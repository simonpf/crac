
% Created 2018-06-25 Mo 13:07
% Intended LaTeX compiler: pdflatex
\documentclass[11pt]{scrartcl}
\usepackage[utf8]{inputenc}
\usepackage[T1]{fontenc}
\usepackage{graphicx}
\usepackage{grffile}
\usepackage{longtable}
\usepackage{wrapfig}
\usepackage{rotating}
\usepackage[normalem]{ulem}
\usepackage{amsmath}
\usepackage{capt-of}
\usepackage{textcomp}
\usepackage{amssymb}
\usepackage{capt-of}
\usepackage{hyperref}
\usepackage{units}
\usepackage[authoryear,round]{natbib}
\author{}
\date{}
\title{Response to reviewer 3}
\hypersetup{
 pdfauthor={Simon Pfreundschuh},
 pdftitle={},
 pdfkeywords={},
 pdfsubject={},
 pdfcreator={Emacs 24.5.1}, 
 pdflang={English}}
\begin{document}

\maketitle

\setlength{\parindent}{0cm}

We thank the referee for the time he/she has put on reading our manuscript and
providing feedback.

Based on the combined comments of the referees, we have decided to implement
these general changes:

\begin{itemize}
\item We will switch to an airborne measurement set-up and
  the introduction section will be modified accordingly
\item The text in the result section will be shortened significantly
\item Redundant results for scene 2 will be placed in an appendix
\item The selection of tested retrieval habits will be revised/changed
\end{itemize}

Below we respond to the main questions raised by the referee, and outline how we
will revise the manuscript.

\section*{Major comments}

\subsection*{Reviewer comment 1}

The paper is presented as an application for ICI in combination with a
Cloudsat like configuration but it is not clear to me what geometry of
observations the authors are thinking about. They state “As mentioned above, the
same incidence angle as for the passive radiometers is assumed also for the
radar. In practice, this could be achieved by remapping the radar observations to
the lines of sights of the passive beam”. Are they thinking about a scanning
W-band radar? or at a off-nadir pointing radar? If the former is true then they
should discuss what is a realistic technological solution (and what are the
consequences in terms of sensitivity) and the authors should refer to state of
the art scanning W-band radar concepts (there is none at the moment!); if the
latter is true they should discuss what are the consequences of such a selection
(e.g. forground clutter) and they need to convince me that what we could gain
from such a configuration compensate from the loss of information introduced by
pointing in such a slanted direction. There should be a certain degree of
“realism” in what we are trying to simulate, especially if this was part of an
ESA study.

\subsubsection*{Author response:}

The reviewer raises a very relevant point with his comment. To address this, we
changed our simulation setup to simulate perfectly co-located observations at
nadir. Realistic modeling of a space-borne viewing geometry (at least in a
variational retrieval) is currently not feasible due to the computational
complexity. We still deem this sufficient for the scope of the study,
i.e. studying the fundamental synergies between active and passive observations.
In addition to this, we follow the recommendation made in the second comment and
will pitch the application more towards air borne observations.

%
%{\itshape 
%In order to reduce the complexity of generating simulated observations, a number
%of simplifications are applied to the viewing geometry and the radiative
%transfer modeling. The beams of all three sensors are assumed to point at nadir
%and to be perfectly coincident pencil beams. In this way, observations for the
%GEM model scenes can be simulated by performing a single 1-dimensional radiative
%transfer calculation for each profile. Moreover, multiple scattering effects in
%the radar observations are neglected. The simulations therefore do not take into
%account beam filling effects caused by atmospheric inhomogeneity across the
%footprints of the different sensors. The incidence angles of the beams of ICI
%and MWI will be around $53^\circ$ at the Earth's surface, so the simulations
%performed here do not represent the viewing geometry of a space-borne
%configuration involving ICI and MWI accurately. Realistic modeling of the
%viewing geometry as well as multiple scattering effects in a variational
%retrieval are currently not feasible at reasonable computational cost with the
%tools used in this study. Since the focus of this study are the fundamental
%synergies between the active and passive observations, this was deemed
%sufficient for its scope. Moreover, these assumptions are justifiable for
%air-borne observations, which adds practical relevance to the simulations.
%}

\subsection*{Reviewer comment 2}

 “the beams of all three sensors are modeled as perfectly coincident pencil
beams”. Again this is quite an assumption. Non uniform beam filling will play a
key factor. This is one of the many simplifications (no polarization, no multiple
scattering,1D, ...) that needs to be clearly listed at the beginning of
Sect.2.2.1 (some appear only at page 27). For this reason I would actually pitch
more towards an airborne configuration where these simplification indeed can
be realistically assumed or of a radar with a radiometric mode (where you can
actually match footprints). Otherwise the (not massive) gain of having a
radar-radiometer combination that you show later on can be completely washed
out by the errors introduced to these assumptions. I imagine that you may also
have airborne data where to test how realistic your forward model is.

\subsubsection*{Author response}

%Also here we fully agree with the points raised by the reviewer. To address
%this, the second paragraph of Sect. 2.1.1 will be revised as presented in the
%previous response. Furthermore, these limitations as well as the applicability
%to airborne observations are now mentioned in the conclusions by adding the
%following lines:

As mentioned above, we will follow the reviewer's suggestion to pitch the application
of the combined retrievals more towards combined retrievals. We will also make these
limitations more clear in Sect. 2.2.1 and discuss their implications more thoroughly.

\subsection*{Reviewer comment 3}
Fig2: these PSDs look very weird to me. Why do they have the plateau at small
sizes? y-axis units are obviously wrong unless you are renormalizing by some mass
(but it is not explained).

\subsubsection*{Author response}

The reviewer is of course right, the units on the axis of the plots were indeed
wrong and will be corrected in the revised manuscript. Otherwise, the PSDs
correspond to the modified-gamma functions that are assumed in the
\cite{milbrandtyau05} microphysics scheme.


\subsection*{Reviewer comment 4}

Fig3: sorry I do not follow what is this (what is the y-axis?), and why this plot is meaningful.

\subsubsection*{Author response}

We will remove this plot from the revised version of the manuscript.


\subsection*{Reviewer comment 5}
Eq.6: Clearly with values lower than 230 K it does not make any sense (negative RH, or large than1.1???)

\subsubsection*{Author response}

We would like to thank the author to point out this inconsistency, as there are indeed two mistakes
in Eq.~6. The right equation should be
\begin{align}
\phi(t) = \begin{cases}
 0.7, & 270\ \unit{K} < t \\
 0.7 + 0.01 \cdot (t - 270), &220 < t \leq  270\ \unit{K} \\
 0.2,  & t < 220 \\
 \end{cases}.
\end{align}
This will of course be corrected in the updated version of the manuscript.

\subsection*{Reviewer comment 6}
Line 210; this means that the vertical resolution changes with the surfacetemperature, really weird choice.

\subsubsection*{Author response}

Indeed, the respective choices for the retrieval grids may not have been optimal.
They will changed to fixed-resolution grids for the revised manuscript.

\subsection*{Reviewer comment 7}

fIG4 : not clear to me why the scattering depression is not increasing at
higher frequencies. I would expect that the optical thickness would drastically
increase increasing frequency. Is this due to very large asymmetry parameters
then? But this is not what I do see in Fig.5 (though Fig4 is of course a very
idealized case) If this is the case then results will be very dependent on
particle habits (which may introduce additional uncertainties in the retrieval)

\subsubsection*{Author response}

It is correct that the scattering extinction increases rapidly with frequency, but the
final scattering depression depends also on other factors. One consideration is the
background absorption due to gases. A higher gas absorption decreases the effect of
scattering, and this effect increases in general with frequency. It is correct that also
the asymmetry parameter needs to be considered, which increases with frequency. A higher
asymmetry parameter gives a lower depression for a given cloud optical depth, see Fig. 5
of \citet{eriksson15}.

It can be a bit hard to judge the scattering depression in a figure like Fig. 5,
as the clear-sky values differ between the channels. In the version found below,
extracted scattering depressions are shown in the second panel. For high-clouds
with moderate cloud optical depth, the scattering depression increases
monotonously with frequency, while in the most dense cloud region (around lat
2.7) this is far from the case for the reasons discussed above.


\begin{figure}[!hbpt]
  \centering
  \includegraphics[width=0.8\textwidth]{../plots/observations_a_3}
  \caption{Simulated brightness temperatures (Panel (a)) and cloud signal
    depressions computed for selected channels of the MWI and ICI radiometers
    for the first test scene.}
  \label{fig:depressions}
\end{figure}

\subsection*{Reviewer comment 8}
8) Line 275: notclear what you mean, in Tab.4 there are 6. 

\subsubsection*{Author response}

What is meant here is that we test different choices for the ice particle shape
which is assumed for the single frozen hydrometeor species which is used in the
retrieval. The particles that are tested are listed in Tab.~4.

Since we will be reformulating the sections describing the selection of the
particle models, this sentence will be reformulated to make it clearer.

\subsection*{Reviewer comment 9}
9) “extends below the sensitivity limit of the passive-only observations around 10-5 kg m-3” : very sloppy sentence. Passive mi-crowave radiometer are sensitive to integrated contents! 

\subsubsection*{Author response}

As response to another reviewer's comment the corresponding paragraph has been
rewritten and the sentence will be removed.

%{\itshape  Panel (c) shows the IWC field retrieved using the passive-only
%  retrieval. Despite a certain resemblance in the overall structure between the
%  retrieved and reference IWC field, the results do not reproduce the vertical
%  structure of the cloud very well. It should be noted, however, that the
%  displayed mass-density range extends below the sensitivity limit of the
%  passive-only observations around $10^{-5}\ \unit{kg\ m^{-3}}$ (c.f. Fig.
%  ~\ref{fig:contours}), which explains the smeared-out appearance of the results
%  to some extent. }


\subsection*{Reviewer comment 10}

 Fig 6d: this retrieval looks really weird. Where are all the stripes coming
 from? Certainly this does not look like acloud, or? What kind of constraint
 have you imposed on the cloud top?

\subsubsection*{Author response}

Yes, in terms of vertical distribution of IWC the passive only retrieval does
not work very well. The reason for this is that the passive observations alone
do not provide much information on the vertical distribution of ice. To correct
for this, further regularization would be necessary which is not applied here in
order to keep the comparison between the different retrieval methods fair. All
of this is discussed in the discussion section of the manuscript.

\subsection*{Reviewer comment 11}

“In general, the radar-only results exhibit only very weak dependency on the
particle model, mak-ing the results for different particle shapes virtually
indistinguishable.” Again another dangerous sentence. We know (unfortunately)
that this is not true (otherwise our ice problems would be sorted). Here my guess
is that you have not properly explored the backscattering variability
(particularly looking at the different degree of riming). It is notclear to me
whether there is enough variability in your ARTS database, I guess you are more
focused at ice particles (including aggregates) but you are not considering
really rimed particles. Regions where graupel is present should be avoided from
the discussion of the radar-only retrieval for the simple reason that in those
regions attenuation correction and multiple scattering effects make the problem
very tricky. I guess that the radiometer as well is in serious trouble when
entering those areas. Again I would not start tackling regions the observation
system is not tailored for.

\subsubsection*{Author response}

We will revise the particle habits used in the retrieval, see below, but we
expect that particle shape will continue to have a smaller impact on our
radar-only retrieval. On the other hand, we see significant errors that can be
referred to particle size distribution (PSD). However, it is difficult in
general to draw a clear line between particle shape and PSD. This is especially
true if particle size is described by Dmax, and the PSD is defined accordingly.
In this case, IWC of a given PSD will depend on the particles effective density,
and e.g. degree of riming becomes critical. Accordingly, to what extent
retrieval errors are due to shape or PSD, depend partly on definitions.

The ARTS single scattering database does include several types of rimed
particles. Two of them are the GemGraupel and GemHail models which are used in
the simulation of the synthetic observations. For the retrieval, however, it is
true that we do not include rimed particles in the tested particle models.

Both the forward simulations and the retrieval handle attenuation consistently.
We therefore think it is worth considering even regions where graupel is present
since in this way it is possible to assess the effect of it not being accurately
represented by the particle model that is used in the retrieval.

It is correct that when inverting real observations multiple scattering needs to be
considered, which adds complexity to the retrieval. We can avoid this extra complexity as
we use simulated observations not including multiple scattering.

\subsection*{Reviewer comment 12}

Fig.10 is missing!!!

\subsubsection*{Author response}

Fig. 10 was unfortunately missing from the manuscript. The figure will be included in the appendix
with the analysis of the second test scene.


\subsection*{Reviewer comment 13}
 “Since the calculation of the AVK involves the forward model Jacobian, this effectmust be related to the non-linearity of the forward model” well I would avoid such veryspeculative statements.

\subsubsection*{Author response}

Following the reviewers suggestion, the sentence will be removed from the manuscript.

\subsection*{Reviewer comment 14}
You need to be very careful how you present the results in Fig. 14. The
conclusions that I can draw is the following: a CloudSat like radar is
pro-viding much more information than the ICI+MWI radiometers when
characterizing ice particles (really the radiometer is providing some additional
water vapour information). As a result we should invest in the former and not the
latter. While I may agree with the previous statement and strongly support a
CloudSat-like radar on an operational mission my feeling is that you are
pitching your radiometer system at the wrong kind of scenes (I already see an
improvement going from the first to the second scene). I would have selected
completely different scenes (including high latitude clouds with mixed phase). It
is to me an overkill to try to retrieve D\_M of rain for these scenes from your
PMW radiometer suite of sensor. If you have any skill in warm rain you
should properly prove it

\subsubsection*{Author response}

We are neither interested in arguing particularly for one or the other
observation system. The question that we want to address is whether the
combination of data has extra value compared to separate observations. The
combination could be achieved by performing joint flights with the aircraft
carrying ISMAR and some other one carrying a radar, or by flying a cloud radar
in constellation to Metop-SG or by adding a sub-millimeter radiometer to the
platform carrying some future cloud radar. That is, to achieve this in practice
could mean no new instrument at all, or either a radar or radiometer.

As the referee clearly favours radars, we would like to balance this by mentioning that
passive instruments have an additional strength in their much higher areal coverage. The
swath of ICI and MWI is about three orders of magnitude broader than the one of CloudSat
and EarthCARE.

Although a cloud radar certainly provides more information on frozen
hydrometeors than ICI, our results clearly show that even radar observations are
insufficient to accurately determine the microphysical properties of ice
hydrometeors (Fig. 4, 7). The passive adds information on the microphysics of
the clouds to the radar (note the significant increase in information content on
$N_0^*$ in Fig. 14) which helps to reduce retrieval uncertainties (Fig. 11).
Although it is not clear whether these improvements carry over to space-borne
observations, our results clearly show this as a synergy between the passive and
active observations (esp. Fig. 4, 11, 14).

The cloud scenes used in the manuscript were selected with the aim of providing
a representative sampling of the type of clouds present in the two model scenes that
were available for the study. We did not want to cherry pick scenes were the
retrieval works well to provide a more realistic assessment of  the retrieval.

Rain must be handled in the retrieval due to its effect on the passive
radiances. However, we never claim that we have any skill in retrieving warm
rain and so we do not agree that we are required to prove to have it.

\subsection*{Reviewer comment 15}

 LWP and Fig.16. I have a serious problem here. The cloud Isee on the right is a
 liquid cloud. So how it is possible that your radiometer is doing sobadly in
 the LWP retrieval and why the combined is so much better? I guess this mustgo
 back to understanding surface emissivity and integrated water vapour (maybe
 somecomments there should be made to explain what kind of surface/IWP we are
 dealingwith). You have not included radar path integrated attenuation in your
 retrieval (like istypically done in radar retrievals) but this could of course
 help in this case.

\subsubsection*{Author response}

The cloud in the right of the scene is a mixed phase cloud. So, considering that
only frequencies from $89\ \unit{GHz}$ and up are included it is not too
surprising that the passive-only configuration doesn't perform well. Again, the
performance of the passive-only retrieval is related to the lack of a priori
information on the vertical position of the cloud. Since liquid water at higher
altitudes has a stronger effect on the observations, the inability of the
passive observations to locate the liquid cloud leads to the observed
underestimation of the water content. This is discussed in Sect. 4.2.3 of the
manuscript.

\subsection*{Reviewer comment 16}

I do not think that for OE to work The forward model must be linear as stated at line 544.

\subsubsection*{Author response}

The OEM can of course be applied to non-linear problems but once complication is
that it can get stuck in secondary minima. The sentence will be corrected in
the revised version of the manuscript.

\subsection*{Reviewer comment 17}
17)Sect.4 and 5: a lot of waffling here (e.g. the three bullet conclusion, you
need to bemuch more quantitative and linked to what you have proved; the three
statements aresomething I could have formulated on my own without making any
simulation). Again the conclusions must be related to the cloud regime you are
considering (and cannot be valid for all!)

\subsubsection*{Author response}

One of the main advantages that we see in the combined retrieval is that it
actually works for a wide range of different cloud regimes. If the cloud regime
was known a priori, good results can probably be achieved using only a radar and
suitable a priori assumptions. In general, however, this is not the case, which
leads to uncertainties in the radar-only retrievals.

For the revised manuscript, we will rewrite the conclusion and parts of the discussion to make
it more concise and the point mentioned above more clear.

\section{Minor comments}

\subsection*{Reviewer comment 1}
I would avoid the use of “ice mass density” and use “ice water content”

\subsubsection*{Author response}

The proposed changes will be adopted in the revised version of the manuscript.

\subsection*{Reviewer comment 2}

Table 2:  it would be good to see footprints as well

\subsubsection*{Author response}

Footprint sizes will be added to the table.

\subsection*{Reviewer comment 3}
Line 130: dBZ are the wrong unitsfor a std of a reflectivity!

\subsubsection*{Author response}
This will be corrected in the revised version of the manuscript.

\subsection*{Reviewer comment 4}

Line 180: “The remaining shape of each PSD is described by the shape parameters
alpha and beta, not to be confused with the parameters of themass-size
relationship shown in Tab. 1.”; very confusing. Why are you using the
sameletters????

\subsubsection*{Author response}

We used the same  letters to be consistent with the definition and used in
\cite{delanoe14} and \cite{cazenave18}.

However, since the explicit values of the $\alpha$ and $\beta$ parameters are
probably of little interest for the average reader, we will simply refer to
\cite{cazenave18} and not name the parameters explicitly.

\subsection*{Reviewer comment 5}
Line 193: wrong units 

\subsubsection*{Author response}

This will be corrected in the revised version of the manuscript.

\subsection*{Reviewer comment 6}

Line 199: English

\subsubsection*{Author response}


This will be corrected in the revised version of the manuscript.

%Although this it was not exactly clear what the comment referred to, the paragraph
%has been revised and now reads as follows:
%
%{\itshape To further regularize the retrieval, $N_0^*$ for ice is retrieved at
%  only 10 equally-spaced grid points between freezing layer and the tropopause.
%  Similarly, $D_m$ and $N_0^*$ for rain are retrieved at 10 respectively 4
%  points between surface and freezing layer. This was necessary for the
%  retrieval to avoid getting stuck in spurious local minima. An approach similar
%  to this one is also taken in the GPM combined precipitation retrievals
%  \citep{grecu16}.}


\subsubsection*{Author response 7}
Line 35 page 2 (not really limited,this is a wide range!!)

\subsubsection*{Author response}

The corresponding sentence will be reformulated in the revised manuscript.

%{\itshape Currently available systems for observing ice hydrometeors typically make use of
%radiation from the microwave, infrared or optical domain.}

\subsection*{Reviewer comment 8}

Line 54 page 2.  maybe it is worth mentioning all the heritage coming from radar-radiometer retrievals with W-band (Ka and Ku-band) radars with PMW radiometers. 

\subsubsection*{Author response}

Following the suggestion of the reviewer a paragraph that mentions previous work
on synergistic retrievals using radar and passive radiometers at lower microwave
frequencies will be added to the introduction.

%{\itshape 
%... has been
%investigated \citep{evans05, jiang19}.
%
%Combined retrievals using radar and passive radiometer observations, have also
%been developed for the Tropical Rainfall Measuring Mission (TRMM,
%\citet{kummerow98, grecu04}) and the Global Precipitation Measurement (GPM,
%\cite{hou14, grecu16, munchak11}) mission. However, since the principal target
%of these missions were liquid hydrometeors, they make use of sensors at
%comparably low microwave frequencies, which provide only limited sensitivity to
%frozen hydrometeors.
%
%This work ...
%}


\subsection*{Reviewer comment 9}
Line 229: “troposphere” is too generic Line

\subsubsection*{Author response}

The use of the word {\itshape troposphere} and should have been {\itshape tropopause}.
This will be corrected in the revised version of the manuscript.

\subsection*{Reviewer comment 10}
Fig 4 caption: you need to include how thick is the layer.

\subsubsection*{Author response}
This will be included in the revised version of the manuscript.

%The figure caption will be revised and now reads:
%
%{\itshape Simulated observations of a homogeneous, $5\ \unit{km}$ thick cloud
%  layer with varying water content $m$ and mass-weighted mean diameter $D_m$.
%  The panels display the maximum radar reflectivity in dBZ overlaid onto the
%  cloud signal measured by selected radiometer channels of the MWI (first row)
%  and ICI radiometers (second row).}

\subsubsection*{Author response 11}
250: rho is not defined

\subsection*{Reviewer comment}

$\rho$ will be defined in the revised version of the manuscript.


\subsection*{Reviewer comment 12}
Line 4: 272.5????

\subsection*{Author response}
This mistake will be corrected in the revised version of the manuscript.

\subsection*{Reviewer comment 13}
Fig 4 caption: you need to include how thick is the layer.

\subsubsection*{Author response}
The layer thickness will be added to the figure caption in the revised version
of the manuscript.

\bibliographystyle{copernicus}
\bibliography{references}

\end{document}
